\chapter{Conclusiones y trabajos futuros}

En capítulos anteriores hemos descrito el contexto y el problema que abordamos en este proyecto, los objetivos iniciales y la solución propuesta junto con una serie de pruebas y experimentos que validad y caracterizan el proyecto. En este capítulo se exponen las conclusiones obtenidas y las posibles líneas por las que se puede continuar el trabajo.\\

\section{Conclusiones}

Bajo una mirada retrospectiva se puede observar que se han cumplido los objetivos generales que nos habíamos marcado. Hemos creado una aplicación web entre navegadores sin servidores intermedios que nos permite controlar, manejar y monitorizar un cuadricóptero. Dentro de este objetivo nos marcamos tres subjetivos, los cuáles también hemos cumplido:\\

\begin{itemize}
\item Hemos creado una conexion local directamente con los sensores y actuadores del drone mediante un navegador web sin necesidad de servidores intermedios.
\item Se ha desarrollado una conexión entre navegadores que transfieren en tiempo real y sin servidores intermedios la cámara y los datos necesarios para monitorar los sensores del drone y teleoperarlo.
\item Se ha creado una interfaz web amigable que nos permite monitorar la cámara y los sensores del drone, así como teleoperarlo de una manera muy intuitiva y sencilla de usar.
\item A parte de estos subobjetivos se le han añadido unos extras, como poder teleoperarlo con un mando o que tanto la conexión local como remota, así como la interfaz sean multiplataforma y multidispositivo, lo que nos permite controlarlo desde un teléfono móvil o tableta, por ejemplo.
\end{itemize} 

Se puede encontrar tanto esta memoria, como el repositorio del código, vídeos, explicaciones, ejemplos y resultados obtenidos en la mediawiki oficial del proyecto\footnote{\url{http://jderobot.org/Irodmar-tfg}}\cite{Mediawiki}.

\section{Trabajos futuros}

A todo proyecto hay que ponerle unos límites, este no es una excepción, por lo que sirve como base y punto de partida para otros proyectos o trabajos con los que ampliar esta aplicación. Dentro de las lineas de desarrollo que se podrían seguir exponemos algunas de ellas:\\

Uno de las líneas más útiles de desarrollo podría ser incorporar diversas cámaras al drone y tener la posibilidad en el par remoto de cambiar entre ellas según nuestras necesidades.\\

Un paso más podría ser dotar al drone de autonomía, pudiendo indicarle desde el ordenador remoto unas coordenadas para que el drone se desplazase de una a otra como si de un circuito se tratase.\\

En definitiva cualquier tipo de funcionalidad que se le pueda añadir a un cuadricóptero se podría implementar en este sistema para que funcionase de manera remota.\\

